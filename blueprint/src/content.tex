\title{InclusionAndExclusion}


\home{https://timechess.github.io/InclusionAndExclusion}
\github{https://github.com/timechess/InclusionAndExclusion}
\dochome{https://timechess.github.io/InclusionAndExclusion/doc}

% \home{localhost:8000}
% \dochome{localhost:8000/doc}

\maketitle


\tableofcontents

\nocite{*} % Delete this line if you have citations.

\section{Introduction}


\section{FinInter}

\begin{definition}\label{List.FinInter}
  \leanok
  \lean{List.FinInter}
  Define the intersection of a list of finite sets indexed by \verb|β|, where each set is given by a function \verb|A : β → Finset α|.
\end{definition}

\begin{lemma}\label{List.eq_FinInter}
  \lean{List.eq_FinInter}
  \uses{List.FinInter}
  \leanok
  Prove that an element is in the intersection of a non-empty list of finite sets if and only if it is in each of the sets.
\end{lemma}

\begin{proof}
  \leanok
  If there's only one element in the list, then it is trival.

  Then we prove by induction.

  Assume the statement hold for a list whose length is $n$, thus an element is in the intersection is in every one of the $n$ sets.

  We add a set $S$ to the beginning of the list. For any element $x$ in the intersection of the $n+1$ sets (which is a subset of the intersection of the previous $n$ sets), it must belongs to the intersection of the previous $n$ sets and $S$. Therefor, $x\in$ the intersection of the $n+1$ sets.

  Thus, the statement holds.

\end{proof}

\begin{definition}\label{Multiset.FinInter}
  \lean{Multiset.FinInter}
  \uses{List.FinInter,List.eq_FinInter}
  \leanok
  Define the intersection of a multiset of finite sets indexed by \verb|β|, where each set is given by a function \verb|A : β → Finset α|.
\end{definition}

\begin{proof}
  \leanok
  To prove that if sets \( A \) and \( B \) are equal, then \( A \cup C = B \cup C \), we can approach the proof by showing two inclusions: \( A \cup C \subseteq B \cup C \) and \( B \cup C \subseteq A \cup C \).

  \textbf{1. Show \( A \cup C \subseteq B \cup C \):}

  \begin{itemize}
    \item Take any element \( x \in A \cup C \).
    \item By the definition of union, \( x \in A \cup C \) means that \( x \) is either in \( A \) or in \( C \).
    \item If \( x \in A \), and since \( A = B \), it follows that \( x \in B \).
    \item Therefore, \( x \in B \cup C \) because \( x \) is in \( B \) (or in \( C \) if it were in \( C \)).
    \item Hence, every element of \( A \cup C \) is also in \( B \cup C \), which proves \( A \cup C \subseteq B \cup C \).
  \end{itemize}

  \textbf{2. Show \( B \cup C \subseteq A \cup C \):}

  \begin{itemize}
    \item Take any element \( x \in B \cup C \).
    \item By the definition of union, \( x \in B \cup C \) means that \( x \) is either in \( B \) or in \( C \).
    \item If \( x \in B \), and since \( A = B \), it follows that \( x \in A \).
    \item Therefore, \( x \in A \cup C \) because \( x \) is in \( A \) (or in \( C \) if it were in \( C \)).
    \item Hence, every element of \( B \cup C \) is also in \( A \cup C \), which proves \( B \cup C \subseteq A \cup C \).
  \end{itemize}

  Since we have shown both inclusions, we conclude that \( A \cup C = B \cup C \).
\end{proof}

\begin{lemma}\label{Multiset.eq_FinInter}
  \lean{Multiset.eq_FinInter}
  \uses{List.FinInter,List.eq_FinInter,Multiset.FinInter}
  \leanok
  Prove that an element is in the intersection of a multiset of finite sets if and only if it is in each of the sets.
\end{lemma}

\begin{proof}
  \leanok
  When talking about the intersection of a multiset of finite sets, we can safely treat the multiset as the list with the same elements in it since order doesn't affect the intersection. So we just need to prove the intersection of a list of finite sets if and only if it is in each of the sets, which is exactly \ref{List.eq_FinInter}.
\end{proof}

\begin{definition}\label{FinInter₀}
  \lean{FinInter₀}
  \uses{Multiset.FinInter}
  \leanok
  Define the intersection of all finite sets indexed by \verb|β|, where each set is given by a function \verb|A : β → Finset α|.
\end{definition}

\begin{lemma}\label{eq_FinInter₀}
  \lean{eq_FinInter₀}
  \uses{Multiset.eq_FinInter,FinInter₀}
  \leanok
  Prove that the intersection of all finite sets indexed by \verb|β| is equal to the intersection of their corresponding sets in \verb|Set α|.
\end{lemma}

\begin{proof}
  \leanok
  By definition, \ref{FinInter₀} is the intersection of a universal multiset of finite sets.
  
  Two sets $A,B$ are equal means for any $x$, $x\in A$ iff $x\in B$.
  
  Using \ref{Multiset.eq_FinInter}, we have for any $m$ in the universal multiset, $x∈A_m$, and the right side is for any $i$ indexed by $β$, $x∈A_i$, which are equivalent.
  
  But \ref{Multiset.eq_FinInter} requires that the universal multiset is not empty.
  
  By the property of the universal multiset, we have for any $t$ indexed by $β$, $t$ is in the universal multiset.
  
  If the universal multiset is empty, then we have a contradiction.

  Thus, the statement holds.
\end{proof}

\section{FinUnion}

\begin{definition}\label{List.FinUnion}
  \leanok
  \lean{List.FinUnion}
  Define the union of a list of finite sets indexed by \verb|β|, where each set is given by a function \verb|A : β → Finset α|.
\end{definition}

\begin{lemma}\label{List.eq_FinUnion}
  \lean{List.eq_FinUnion}
  \uses{List.FinUnion}
  Prove that an element is in the union of a list of finite sets if and only if it is in at least one of the sets.
\end{lemma}

\begin{proof} 
  \leanok
  Consider that we define union of sets by induction , we enumeration pointer until a set contain the element.So obviously ,they are same.
\end{proof}

\begin{definition}\label{Multiset.FinUnion}
  \lean{Multiset.FinUnion}
  \uses{List.FinUnion,List.eq_FinUnion}
  Define the union of a multiset of finite sets indexed by \verb|β|, where each set is given by a function \verb|A : β → Finset α|.
\end{definition}

\begin{lemma}\label{Multiset.eq_FinUnion}
  \lean{Multiset.eq_FinUnion}
  \uses{List.FinUnion,List.eq_FinUnion,Multiset.FinUnion}
  Prove that an element is in the union of a multiset of finite sets if and only if it is in at least one of the sets.
\end{lemma}

\begin{proof} 
  \leanok
  Multiset just have some repeat element. So it's also obviously when List.eq_FinUnion is right
\end{proof}

\begin{definition}\label{FinUnion₀}
  \lean{FinUnion₀}
  \uses{Multiset.FinUnion}
  Define the union of all finite sets indexed by \verb|β|, where each set is given by a function \verb|A : β → Finset α|.
\end{definition}

\begin{lemma}\label{eq_FinUnion₀}
  \lean{eq_FinUnion₀}
  \uses{Multiset.eq_FinUnion,FinUnion₀}
  Prove that the union of all finite sets indexed by \verb|β| is equal to the union of their corresponding sets in \verb|Set α|.
\end{lemma}

\begin{proof}
    \leanok
    It's also really obviously when Multiset.eq_FinUnion is right.
\end{proof}
\section{ToInt}

\begin{definition}\label{toInt}
  \leanok
  \lean{toInt}
  Assign a value to a proposition. If the proposition holds, assign a value of 1; otherwise, assign a value of 0.
\end{definition}

\begin{lemma}\label{toInt_and}
  \lean{toInt_and}
  \uses{toInt}
  \leanok
  The value of both propositions P and Q holding is equal to the value of P times the value of Q.
\end{lemma}

\begin{proof}
  \leanok
\end{proof}

\begin{lemma}\label{toInt_not}
  \lean{toInt_not}
  \uses{toInt}
  \leanok
  The value of the negation of a proposition P is equal to 1 minus the value of P.
\end{lemma}

\begin{proof}
  \leanok
\end{proof}

\begin{definition}\label{char_fun}
  \leanok
  \lean{char_fun}
  \uses{toInt}
  Define a characteristic function that returns 1 if \verb|x| is in the finite set \verb|A|, and 0 otherwise.
\end{definition}

\begin{lemma}\label{char_fun_inter}
  \lean{char_fun_inter}
  \uses{toInt,toInt_and,char_fun}
  The characteristic function of the intersection of two finite sets \verb|A| and \verb|B| is equal to the product of their characteristic functions.
\end{lemma}

\begin{lemma}\label{char_fun_union}
  \lean{char_fun_union}
  \uses{toInt,toInt_and,toInt_not,char_fun}
  The characteristic function of the union of two finite sets \verb|A| and \verb|B| is equal to 1 minus the product of 1 minus their characteristic functions.
\end{lemma}

\begin{proof}
    \leanok

Consider a family of finite sets \(\{A_\beta\}\) indexed by \(\beta\), and let the union of these sets be \(A = \bigcup_\beta A_\beta\). The characteristic function \(\chi_{A_\beta}\) takes the value 1 for elements in \(A_\beta\) and 0 for elements outside.

For the characteristic function \(\chi_A\) of the union \(A\), it takes the value 1 for any element in \(\bigcup_\beta A_\beta\) and 0 for elements outside the union. Our goal is to show that \(\chi_A = 1 - \prod_\beta (1 - \chi_{A_\beta})\).

Let's reason as follows: for any element outside the union \(A\), it is not in any of the \(A_\beta\), so \(\chi_{A_\beta}(x) = 0\) for every \(\beta\), making \((1 - \chi_{A_\beta}(x)) = 1\). Therefore, the product \(\prod_\beta (1 - \chi_{A_\beta}(x)) = 1\), and thus \(1 - \prod_\beta (1 - \chi_{A_\beta}(x)) = 0\), matching the value of \(\chi_A\) outside the union.

On the other hand, for any element inside the union \(A\), it belongs to at least one set \(A_\beta\). This means there exists at least one \(\beta\) such that \(\chi_{A_\beta}(x) = 1\). Consequently, there is at least one \((1 - \chi_{A_\beta}(x)) = 0\). The product \(\prod_\beta (1 - \chi_{A_\beta}(x))\) hits zero as soon as the first zero is encountered, and thus \(1 - \prod_\beta (1 - \chi_{A_\beta}(x)) = 1\), matching the value of \(\chi_A\) inside the union.。

\end{proof}

\section{Auxiliary}

\begin{definition}\label{Finset.powerset₀}
  \leanok
  \lean{Finset.powerset₀}
  Define the set of all nonempty subsets of a given finite set \verb|A|.
\end{definition}

\begin{lemma}\label{card_eq}
  \lean{card_eq}
  \leanok
  If a finite set \verb|A| is equal to a set \verb|B| in the sense of set theory, then they have the same number of elements.
\end{lemma}

\begin{lemma}\label{mul_expand₃}
  \lean{mul_expand₃}
  \leanok
  Formalize the polynomial expansion of the product \(\prod_{i=0}^{n-1} (1 - g(i))\).
\end{lemma}

\begin{lemma}\label{mul_expand₂}
  \lean{mul_expand₂}
  \leanok
  \uses{Finset.powerset₀,mul_expand₃}
  Formalize the polynomial expansion of \(1 - \prod_{i=0}^{n-1} (1 - g(i))\).
\end{lemma}

\begin{lemma}\label{mul_expand₁}
  \lean{mul_expand₁}
  \leanok
  \uses{Finset.powerset₀,mul_expand₂}
  Formalize the polynomial expansion of \(1 - \prod_{i=0}^{n-1} (1 - g(i))\) in the context of finite sets of natural numbers up to \verb|n|.
\end{lemma}


\section{MainTheorem}

\begin{lemma}\label{card_eq_sum_char_fun}
  \lean{card_eq_sum_char_fun}
  \uses{toInt,char_fun,card_eq}
  \leanok
  The cardinality of a subset \verb|B| of \verb|A| is equal to the sum of the characteristic function of \verb|B| over \verb|A|.
\end{lemma}

\begin{proof}
  \leanok
  The right-hand side of the equation is equivalent to the number of elements in the intersection of sets $A$ and $B$. The left-hand side of the equation represents the number of elements in set $B$. Since $B$ is a subset of $A$, the intersection of $A$ and $B$ is equal to $B$. Therefore, the equality holds.

  Mathematically, this can be stated as:
  \begin{equation*}
  |B| = |A \cap B|
  \end{equation*}
  given that $B \subseteq A$.
\end{proof}

\begin{lemma}\label{card_eq_FinInter}
  \lean{card_eq_FinInter}
  \uses{FinInter₀,eq_FinInter₀,card_eq}
  The cardinality of the intersection of a family of finite sets indexed by \verb|β| is equal to the cardinality of their intersection in \verb|Finset| form.
\end{lemma}

\begin{lemma}\label{card_eq_FinUnion}
  \lean{card_eq_FinUnion}
  \uses{card_eq,FinUnion₀,eq_FinUnion₀}
  The cardinality of the union of a family of finite sets indexed by \verb|β| is equal to the cardinality of their union in \verb|Finset| form.
\end{lemma}

\begin{proof}
    \leanok
The cardinality of the union of a family of finite sets indexed by \(\beta\) is equal to the cardinality of their union in \(\text{Finset}\) form. Here, the characteristic function represents the size of the set. Let's provide a proof in plain language.

Consider a family of finite sets \(\{A_\beta\}\) indexed by \(\beta\), and let the union of these sets be \(A = \bigcup_\beta A_\beta\). We represent the union of these sets in \(\text{Finset}\) form, ensuring that all set elements are unique. In \(\text{Finset}\) form, the cardinality of a set is simply the number of elements in the set.

To show that the cardinality of \(A\) is equal to the cardinality of the union in \(\text{Finset}\) form, we compare the elements in both sets. Any element in \(A\) must belong to at least one set \(A_\beta\). Since \(\text{Finset}\) guarantees the uniqueness of elements, this means that in the union in \(\text{Finset}\) form, each element also occurs exactly once.

Thus, regardless of whether an element belongs to a single set or appears in multiple sets, as long as it is in the union \(A\), it will also appear once in the union in \(\text{Finset}\) form. This ensures that the cardinality of \(A\) is precisely equal to the cardinality of the union in \(\text{Finset}\) form.

\end{proof}

\begin{lemma}\label{char_fun_FinInter}
  \lean{char_fun_FinInter}
  \uses{FinInter₀,eq_FinInter₀,toInt,char_fun}
  \leanok
  The characteristic function of the intersection of a family of finite sets indexed by \verb|β| is equal to the product of their characteristic functions.
\end{lemma}

\begin{proof}
  \leanok
  Let $\mathcal{F} = \{A_1, A_2, \ldots, A_n\}$ be a collection of sets and let $B = \bigcap_{A \in \mathcal{F}} A$ be the intersection of these sets. Consider the characteristic function $\chi_B(x)$ for any element $x$. We wish to establish the following equivalence:

  \begin{equation*}
  \chi_B(x) = \prod_{A \in \mathcal{F}} \chi_A(x)
  \end{equation*}

  We analyze two cases:

  Case 1: $x$ belongs to the intersection $B$

  \textbf{Left Hand Side (LHS):} $\chi_B(x) = 1$. This condition is equivalent to $x$ belonging to every set $A \in \mathcal{F}$.

  \textbf{Right Hand Side (RHS):} The product becomes $\prod_{A \in \mathcal{F}} \chi_A(x) = \prod_{A \in \mathcal{F}} 1 = 1$. Therefore, LHS = RHS.

  Case 2: $x$ does not belong to the intersection $B$

  \textbf{Left Hand Side (LHS):} $\chi_B(x) = 0$. This is equivalent to the existence of a set $A \in \mathcal{F}$ such that $x \not\in A$.

  \textbf{Right Hand Side (RHS):} There exists an $A \in \mathcal{F}$ for which $\chi_A(x) = 0$. Since the product involves a zero, $\prod_{A \in \mathcal{F}} \chi_A(x) = 0$. Thus, LHS = RHS.
  In conclusion, for any element $x$, we have $\chi_B(x) = \prod_{A \in \mathcal{F}} \chi_A(x)$, which completes the proof.

\end{proof}

\begin{lemma}\label{char_fun_FinUnion}
  \lean{char_fun_FinUnion}
  \uses{FinUnion₀,eq_FinUnion₀,toInt,char_fun}
  The characteristic function of the union of a family of finite sets indexed by \verb|β| is equal to 1 minus the product of 1 minus their characteristic functions.
\end{lemma}

\begin{proof}
  \leanok
The characteristic function of the union of a family of finite sets indexed by \(\beta\) is equal to 1 minus the product of 1 minus their characteristic functions. Let's prove this theorem using plain language.

Suppose we have a family of finite sets indexed by \(\beta\), denoted as \(\{A_\beta\}\), and consider the union of these sets, \(A = \bigcup_\beta A_\beta\). The characteristic function \(\chi_{A_\beta}\) defines a function for each set \(\beta\), being 1 for any element in \(A_\beta\) and 0 for any element not in the set.

Now, consider the characteristic function \(\chi_A\) of the union \(A\), which also follows the rule: 1 for elements in \(A\) and 0 for elements not in \(A\).

To prove that \(\chi_A = 1 - \prod_\beta (1 - \chi_{A_\beta})\), we consider two scenarios:

1. **Elements not in the union** (\(x \not\in A\)): If an element is not in \(A\), it means it is not in any \(A_\beta\). Thus, for every characteristic function \(\chi_{A_\beta}\) of a set \(A_\beta\), the value of the function for that element is 0. Hence, \(1 - \chi_{A_\beta} = 1\). Since the element is not in any \(A_\beta\), \(\prod_\beta (1 - \chi_{A_\beta}) = 1\), and thus \(1 - \prod_\beta (1 - \chi_{A_\beta}) = 0\), which aligns with the definition of the characteristic function \(\chi_A\) (value 0 for elements not in the union).

2. **Elements in the union** (\(x \in A\)): If an element is in \(A\), the element is in at least one \(A_\beta\). This means there is at least one \(\beta\) for which \(\chi_{A_\beta} = 1\). Thus, there is at least one \(1 - \chi_{A_\beta} = 0\). The product \(\prod_\beta (1 - \chi_{A_\beta})\) will be zero due to encountering at least one 0, leading to \(1 - \prod_\beta (1 - \chi_{A_\beta}) = 1\). This also aligns with the definition of the characteristic function \(\chi_A\) (value 1 for elements in the union).

With this analysis, we have proven the theorem.
\end{proof}

\begin{lemma}\label{mul_expand₀}
  \lean{mul_expand₀}
  \uses{Finset.powerset₀,mul_expand₁}
  \leanok
  The polynomial expansion of \(1 - \prod_{i=0}^{n-1} (1 - g(i))\).
\end{lemma}

\begin{proof}
  \leanok
  We construct a function \( g' : \mathbb{N} \to \mathbb{Z} \). When x < n, g'(x) is defined as g(x), when x >= n ,g'(x) is defined ad 0.

  Since we only use the x < n part of g(x) in the statement, it is obvious that we can change g to g' in the statement.

  Now the statement is completely the same as \ref{mul_expand₁}. 
\end{proof}

\begin{theorem}\label{Principle_of_Inclusion_Exclusion}
  \lean{Principle_of_Inclusion_Exclusion}
  \uses{FinInter₀,FinUnion₀,eq_FinInter₀,eq_FinUnion₀,Finset.powerset₀,char_fun,card_eq_sum_char_fun,char_fun_FinInter,char_fun_FinUnion,card_eq,card_eq_FinInter,card_eq_FinUnion,mul_expand₀}
  The principle of inclusion-exclusion for a finite union of finite sets.
\end{theorem}

