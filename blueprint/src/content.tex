\title{InclusionAndExclusion}


\home{https://timechess.github.io/InclusionAndExclusion/}
\github{https://github.com/timechess/InclusionAndExclusion/}
\dochome{https://timechess.github.io/InclusionAndExclusion/doc/}

% \home{localhost:8000}
% \dochome{localhost:8000/doc}

\maketitle


\tableofcontents

\nocite{*} % Delete this line if you have citations.

\section{Introduction}


\section{FinInter}

\begin{definition}\label{List.FinInter}
  \leanok
  \lean{List.FinInter}
          Given finite number of finite sets, List.FinInter returns their intersection using an inductive way.
\end{definition}

\begin{lemma}\label{List.eq_FinInter}
  \lean{List.eq_FinInter}
  \uses{List.FinInter}

\end{lemma}

\begin{definition}\label{Multiset.FinInter}
  \lean{Multiset.FinInter}
  \uses{List.FinInter,List.eq_FinInter}
          We say the intersection of several finite sets does not depend on the order in which who and who intersect first. Therefore, we introduce the definition of multiset. Given a function A, we define a new function from a multiset to the intersection of finite sets whose index is in the multiset
\end{definition}

\begin{lemma}\label{Multiset.eq_FinInter}
  \lean{Multiset.eq_FinInter}
  \uses{List.FinInter,List.eq_FinInter,Multiset.FinInter}
\end{lemma}

\begin{definition}\label{FinInter₀}
  \lean{FinInter₀}
  \uses{Multiset.FinInter}

\end{definition}

\begin{lemma}\label{eq_FinInter₀}
  \lean{eq_FinInter₀}
  \uses{Multiset.eq_FinInter,FinInter₀}

\end{lemma}

% To avoid bibtex errors
\section{FinUnion}

\begin{definition}\label{List.FinUnion}
  \leanok
  \lean{List.FinUnion}
          Given finite number of finite sets, List.FinInter returns their union using an inductive way
\end{definition}

\begin{lemma}\label{List.eq_FinUnion}
  \lean{List.eq_FinUnion}
  \uses{List.FinUnion}

\end{lemma}

\begin{definition}\label{Multiset.FinUnion}
  \lean{Multiset.FinUnion}
  \uses{List.FinUnion,List.eq_FinUnion}
          We define a new function from a multiset to the union of finite sets whose index is in the multiset
\end{definition}

\begin{lemma}\label{Multiset.eq_FinUnion}
  \lean{Multiset.eq_FinUnion}
  \uses{List.FinUnion,List.eq_FinUnion,Multiset.FinUnion}
\end{lemma}

\begin{definition}\label{FinUnion₀}
  \lean{FinUnion₀}
  \uses{Multiset.FinUnion}

\end{definition}

\begin{lemma}\label{eq_FinUnion₀}
  \lean{eq_FinUnion₀}
  \uses{Multiset.eq_FinUnion,FinUnion₀}

\end{lemma}

\section{ToInt}

\begin{definition}\label{toInt}
  \leanok
  \lean{toInt}
          We assign a value to a proposition. If the proposition holds, we assign a value of 1; otherwise, we assign a value of 0
\end{definition}

\begin{lemma}\label{toInt_and}
  \lean{toInt_and}
  \uses{toInt}
          The value of P and Q both holds is equal to the value of P times the value of Q
\end{lemma}

\begin{lemma}\label{toInt_not}
  \lean{toInt_not}
  \uses{toInt}
          The value of not P is equal to one sub the value of P
\end{lemma}

\begin{definition}\label{char_fun}
  \leanok
  \lean{char_fun}
  \uses{toInt}
          We define a function that if x in A then returns 1, else returns 0
\end{definition}

\begin{lemma}\label{char_fun_inter}
  \lean{char_fun_inter}
  \uses{toInt,toInt_and,char_fun}
          We claim that x in the intersection of A and B is equal to x in A and x in B both holds
\end{lemma}

\begin{lemma}\label{char_fun_union}
  \lean{char_fun_union}
  \uses{toInt,toInt_and,toInt_not,char_fun}
          We claim that x in the union of A and B is equal to at least one of x in A and x in B holds
\end{lemma}
\section{Auxiliary}

\begin{definition}\label{Finset.powerset₀}
  \leanok
  \lean{Finset.powerset₀}
          We define all the nonempty subsets of A to be A.
\end{definition}

\begin{lemma}\label{card_eq}
  \lean{card_eq}
          It's obvious that if a Finset A equals to a Set B in the view of Set, then they have the same number of elements. Since we use this lemma a lot in below proof, we put it here to be an independent lemma
\end{lemma}

\begin{lemma}\label{mul_expand₃}
  \lean{mul_expand₃}

\end{lemma}

\begin{lemma}\label{mul_expand₂}
  \lean{mul_expand₂}
  \uses{Finset.powerset₀,mul_expand₃}

\end{lemma}

\begin{lemma}\label{mul_expand₁}
  \lean{mul_expand₁}
  \uses{Finset.powerset₀,mul_expand₂}

\end{lemma}


\section{MainTheorem}

\begin{lemma}\label{card_eq_sum_char_fun}
  \lean{card_eq_sum_char_fun}
  \uses{toInt,char_fun,card_eq}

\end{lemma}

\begin{lemma}\label{card_eq_FinInter}
  \lean{card_eq_FinInter}
  \uses{FinInter₀,eq_FinInter₀,card_eq}
\end{lemma}

\begin{lemma}\label{card_eq_FinUnion}
  \lean{card_eq_FinUnion}
  \uses{card_eq,FinUnion₀,eq_FinUnion₀}
\end{lemma}

\begin{lemma}\label{char_fun_FinInter}
  \lean{char_fun_FinInter}
  \uses{FinInter₀,eq_FinInter₀,toInt,char_fun}

\end{lemma}

\begin{lemma}\label{char_fun_FinUnion}
  \lean{char_fun_FinUnion}
  \uses{FinUnion₀,eq_FinUnion₀,toInt,char_fun}

\end{lemma}

\begin{lemma}\label{mul_expand₀}
  \lean{mul_expand₀}
  \uses{Finset.powerset₀,mul_expand₁}

\end{lemma}

\begin{theorem}\label{Principle_of_Inclusion_Exclusion}
  \lean{Principle_of_Inclusion_Exclusion}
  \uses{FinInter₀,FinUnion₀,eq_FinInter₀,eq_FinUnion₀,Finset.powerset₀,char_fun,card_eq_sum_char_fun,char_fun_FinInter,char_fun_FinUnion,card_eq,card_eq_FinInter,card_eq_FinUnion,mul_expand₀}
          Finally, we can start to formalize the main theorem.
\end{theorem}

